\documentclass[a4paper,10pt]{article}
\usepackage[utf8x]{inputenc}

%opening
\title{Proof}
\author{Shayan Najd}

\begin{document}
\section{Proof}
A complete CFG of a functional language, $CCFG = (N,E)$ with $N$ being a set of nodes and $E$ a set of ordered pairs of nodes representing edges, has the following properties:

\begin{enumerate}
 \item Acyclic, since there is no loop structure
 \item Directed
 \item Connected
 \item Binary,$\forall n \in N, 0\leq deg^+(n)\leq2 $ . Since the decisions are boolean, there are maximum two outgoing edges possible for each node.
 \item Rooted,$|\{n\in N |deg^-(n)=0\} | = 1$ . There is one single root for the graph, the \emph{source node}, and it has only one outgoing edge.
 \item $|\{n\in N |deg^+(n)=0\} | = 1$; there is one \emph{sink node}.
 \item Each edge leaving a node $n \in N$ such $deg^+(n)=2$, is connected to a node that has exactly one incoming edge. In other words, each branch starts with a non-joint node which is a node with one single incoming edge.
\end{enumerate}

To prove "if every node $n \in N$ is visited then all the edges $e \in E$ are visited", we first prove that every outgoing edge is visited. Then, by cases, we prove that all the ingoing edges are visited.

For every node $n \in N$ that is not the \emph{source node} nor the \emph{sink node}, with one outgoing edge $e$:
\begin{enumerate}
 \item If every node is visited then $n$ is visited 
 \item Since the CFG is connected and $n$ is not the \emph{sink node}, If the execution flow reaches $n$ then it should exit from the only edge $e$
 \item If $n$ is visited then $e$ is visited
\end{enumerate}

For every node $n,n1,n2 \in N$ that is not the \emph{source node} or the \emph{sink node}, that $n$ has two outgoing edges $e1=(n,n1)$ and $e2=(n,n2)$:
\begin{enumerate}
 \item If every node is visited then $n$,$n1$ and $n2$ are all visited 
 \item According to the property $(7)$, since $deg^+(n)=2$, We have $deg^-(n1)=1$ and $deg^-(n2)=1$. Hence, $e1$ is the only way to reach $n1$ and $e2$ is the only way to each $n2$
 \item If $n1$ is visited then $e1$ is visited.
 \item If $n2$ is visited then $e2$ is visited.
 \item If $n$, $n1$ and $n2$ are all visited then $e1$ and $e2$ are visited
\end{enumerate}

Since the \emph{source node} has only one outgoing edge, if it is visited then its outgoing edges are visited too. The \emph{sink node} has no outgoing edge.

Thus, if every node is visited then all the respective outgoing edges are visited. Since the set of all the outgoing edges is equal to $E$, if every node $n \in N$ is visited then all the edges $e \in E$ are visited.



\end{document}
